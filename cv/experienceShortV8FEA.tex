%-------------------------------------------------------------------------------
%	SECTION TITLE
%-------------------------------------------------------------------------------
\cvsection{Experience}


%-------------------------------------------------------------------------------
%	CONTENT
%-------------------------------------------------------------------------------
\begin{cventries}

%---------------------------------------------------------
  \cventryprojtrueb
    {National Institute of Aerospace and NASA LaRC} % Organization  
    {Hampton, VA, USA} % Location
    {Research Scholar} % Job title
    {Nov. 2015 - PRESENT} % Date(s)
    {Project 1} %{\textbf{Project 1}.}
    {{Damage scattered wave extraction for large-scale complex composite aerospace structures.} (The only NIA Internal Research and Development (IRAD) funding granted for 2016. \bfseries{Program PI: Dr. Jiaze He})}
    {     
     \begin{cvitems} % Description(s) of tasks/responsibilities
       \item {Independently developed multiple implementations of Radon transform for recognition and extraction of damage scattered signals in complex structures.}
      \end{cvitems}
         }
  \cventryprojtruec
   {Project 2} %{\textbf{Project 1}.}
   {{Enabling the Digital Twin by the Development of Smart Hangar Technology.} (NASA Grant\ /Cooperative Agreement Number: \#NNX12AL15A. Program PI: Dr. William Leser, william.leser@nasa.gov.)}
       % {\underline{Damage scattered wave extraction for large-scale complex composite aerospace structures.} (The only NIA Internal Research and Development (IRAD) funding granted for 2016. \bfseries{Program PI: Dr. Jiaze He})}
    {     
     \begin{cvitems} % Description(s) of tasks/responsibilities
       \item {Developed a high-resolution imaging algorithm, least-squares reverse-time migration (LSRTM) using a conjugate gradient descent optimization method (coded by myself), collaborating with the C.H. Green Chair of Center for Wave Phenomena at Colorado School of Mines. }
       \item {Developed a multi-mode imaging algorithm for complex aerospace structures, teamed up with Dr. Cara Leckey of NASA Langley's Non-Destructive Evaluation Branch.}
       \item {Applied multiple machine learning algorithms (Multivariate Regression,PCA, deep learning using TensorFlow) to generate a surrogate model, achieving near real time inspection in the digital twin development.}
       \item {Developed advanced fatigue damage detection and imaging techniques for complex aerospace structures.}
      \end{cvitems}
         }

%---------------------------------------------------------
 \cventryprojtrued
    {North Carolina State University} % Organization  
    {Raleigh, NC, USA} % Location
    {Adjunct Assistant Professor} % Job title
    {Apr. 2016 - PRESENT} % Date(s)
    {     
     \begin{cvitems} % Description(s) of tasks/responsibilities
       \item {Was invited for writing a comprehensive review paper in the field of wavefield analyses for a special issue in JIMSS journal.}
       \item {Contributed efforts to the University Leadership Initiative (ULI) Project from NASA.}
      \end{cvitems}
         }

  \cventryprojtrueb
    {National Institute of Aerospace} % Organization  
    {Hampton, VA, USA} % Location
    {Graduate Research Assistant} % Job title
    {Oct. 2014 - Oct. 2015} % Date(s)
    {Project 3} %{\textbf{Project 1}.}
    {{Automated Damage Location and Quantitative Level of Damage Algorithms for Autonomous Vehicle Health Monitoring} (Under the NASA Vehicle Systems Safety Technologies project. NASA Grant\ /Cooperative Agreement Number: \#NNL009A00A. Program manager: Mr. Richard Ross, richard.w.ross@nasa.gov.)}
    {     
     \begin{cvitems} % Description(s) of tasks/responsibilities
       \item {Developed a \textbf{time-reversal multiple signal classification (TR-MUSIC)} algorithm with sub-wavelength imaging capability suitable for small defect detection, i.e. corrosion. }
       \item {Optimized models of airplane wings using 3D CAD modeling in SolidWorks and FEA.}
       \item {Analyzed pre-stressed structures using ANSYS to study the nonlinear problem.}
       \item {Used AutoCAD to design mechanical drawings for varying applications.}
      \end{cvitems}
         }

%---------------------------------------------------------
%  \cventryprojtruec
%    {Project 3} %{\textbf{Project 1}.}
%   {{Automated Damage Location and Quantitative Level of Damage Algorithms for Autonomous Vehicle  
%          Health Monitoring} (Under the NASA Vehicle Systems Safety Technologies project. NASA Grant\ /Cooperative Agreement Number: \#NNL009A00A. Program manager: Mr. Richard Ross, richard.w.ross@nasa.gov.)}
       % {\underline{Damage scattered wave extraction for large-scale complex composite aerospace structures.} (The only NIA Internal Research and Development (IRAD) funding granted for 2016. \bfseries{Program PI: Dr. Jiaze He})}
%    {     
%     \begin{cvitems} % Description(s) of tasks/responsibilities
%       \item {Used wave-based methods and image processing technologies to autonomously monitor %vehicle health.}
%       \item {Developed \textbf{reverse-time migration (RTM)} and \textbf{DORT} imaging techniques for damage imaging in a C-17 composite aileron for the first time.}
%       \item {Developed a \textbf{time-reversal multiple signal classification (TR-MUSIC)} algorithm with sub-wavelength imaging capability suitable for small defect detection, i.e. corrosion. }
%       \item {Developed a non-contact system with a limited number of data acquisition points using a laser Doppler vibrometer.}
%       \item {Wrote four journal papers and three conference papers from this project.}
%      \end{cvitems}
%         }

%---------------------------------------------------------
 \cventryprojtrued
    {Bose Corporation - ElectroForce Group} % Organization  
    {Minneapolis, MN, USA} % Location
    {Engineering Intern} % Job title
    {May 2014 - Aug. 2014} % Date(s)
    {     
     \begin{cvitems} % Description(s) of tasks/responsibilities
       \item {Designed and performed uniaxial fatigue tests on rubbers and 3-point bending fatigue tests on composites.}
       \item {Designed and performed dynamic mechanical analysis for shape memory alloys, rubbers, and composites with extreme temperature changes.}
       \item {Obtained the best DMA curves achieved using the ElectroForce 3200 in the company history, which were selected for marketing materials afterwards.}
       \item {Communicated with Tech Sales Support and customers in China in Mandarin directly.}       
      \end{cvitems}
         }

%---------------------------------------------------------
  \cventryprojtrueb
    {Smart Structures \& Materials Lab, North Carolina State University} % Organization  
    {Raleigh, NC, USA} % Location
    {Research Assistant} % Job title
    {Aug. 2011 - Aug. 2014} % Date(s)
    {Project 4} %{\textbf{Project 1}.}
    {{Bayesian Framework Based Damage Segmentation (BFDS) with Time-Reversal Tomography (TRT) for Damage Characterization in Complex Aircraft Structures.} (NASA STTR \# NNX13CD10P). Worked with NASA Dryden Flight Center and X-wave Innovations, Inc. (XII).}
    {     
     \begin{cvitems} % Description(s) of tasks/responsibilities
       \item {Developed a Bayesian image segmentation algorithm with Markov Random Field Priors for damage sizing, which allowed for estimation of the probability of detection (POD) and confidence levels (CL).}
       \item {Designed and developed an imaging system prototype using diffraction tomography to image damage in composite plates.}
       \item {Studied various data processing and damage identification methods in NDE and medical imaging.}
      \end{cvitems}
         }
  \cventryprojtruec
    {Project 5} %{\textbf{Project 1}.}
   {{A Portable, Linear-Array Ultrasonic Imaging System for Rapid Inspection of Large-Area Composite Structures.} (NASA SBIR \#NNX12CF03P). Worked with Kennedy Space Center and X-wave Innovations, Inc. (XII).}
    {     
     \begin{cvitems} % Description(s) of tasks/responsibilities
       \item {Developed a reverse-time migration (RTM) algorithm for damage imaging in composites. }
       \item {Designed and tested a portable transducer array for large plate-like structures.}
       \item {Investigated the Lamb wave velocity change induced by the nonlinearity effect in prestressed plates.}
      \end{cvitems}
         }
  \cventryprojtruec
    {Project 6} %{\textbf{Project 1}.}
   {{Development of a Self-sustained Wireless Integrated Structural Health Monitoring System for Highway Bridges.} Worked with Univ. of Maryland and Department of Transportation (DOT). }
    {     
     \begin{cvitems} % Description(s) of tasks/responsibilities
       \item {Utilized an iterative retrieval technique for acoustic emission signals using a time-reversal method.}
       \item {Monitored a swing bridge remotely via wind and solar powered sensor networks.}
       \item {Served as a group leader to cooperate with researchers from UMD and technicians from NCDOT.}
      \end{cvitems}
         }


%---------------------------------------------------------
 \cventryprojtrued
    {Center of Structural Monitoring and Control, Harbin Institute of Technology} % Organization  
    {Harbin, China} % Location
    {Visiting Scholar} % Job title
    {Oct. 2010 - Jun. 2011} % Date(s)
    {     
     \begin{cvitems} % Description(s) of tasks/responsibilities
       \item {Studied fatigue damage identification and accumulation using acoustic emission (AE) techniques and observed damage patterns using a scanning electron microscope (SEM).}
       \item {Manufactured basalt fiber reinforced polymer (FRP) and performed three-point bending tests.}
     \end{cvitems}
         }

%---------------------------------------------------------
 \cventryprojtrued
    {Department of Materials Science \& Engineering, North Carolina State University} % Organization  
    {Raleigh, NC, USA} % Location
    {Undergraduate Intern} % Job title
    {Jul. 2010 - Aug. 2010} % Date(s)
    {     
     \begin{cvitems} % Description(s) of tasks/responsibilities
       \item {Under the direction of Distinguished Professor Yuntian T. Zhu}
       \item {Synthesized carbon nanotubes (CNTs) via chemical vapor deposition (CVD), spanned CNTs into carbon nanotube fibers (CNFs), and tested the piezoresistive effect.}
     \end{cvitems}
         }


%---------------------------------------------------------
\end{cventries}
